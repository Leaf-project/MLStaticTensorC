\documentclass[11pt]{article}
    \title{\textbf{Math language}}
    \author{Kashtanov Vadim}
    \date{}
    
    \addtolength{\topmargin}{-3cm}
    \addtolength{\textheight}{3cm}
\begin{document}

\maketitle
\thispagestyle{empty}

\section{Equations}
If we set an equation \[y = exp(4x + 1)\] We can see some variables and any-circonstance global constants : 4, x, 1 and the function exp. We can rewrite it as \[\alpha=\{4,x,1\}\] \[y = exp(\alpha_{0} * \alpha_1 + \alpha_2)\]
In fact an eqation is a representation of relatingchip between argument's throught operation, where functions are just macros (ex: $\exp(x) = e^{x}$).

\section{Derivatives}
If we take equation as $y = x^{x}$ it's partial derivative in x will be \[\frac{\partial y}{\partial x} = x^{x}(ln(x)+1)\]
But if we rewrite the equation with $\alpha = \{x,x_1\}$ where $x_1 = x$ and $y_{1} = y$ then we can write
\[y_{1} = x^{x_{1}}\]
then
\[\frac{\partial y_{1}}{\partial x} = x_{1}x^{x_{1}-1} = x^{x_{1}} \ (because\ x=x_{1})\]
\[\frac{\partial y_{1}}{\partial x_{1}} = x^{x_{1}} ln (x_{1})\]
then

\[
\left \{
\begin{array}{r c l}
	y = x^{x} \\
	y_1 = x^{x_{1}} \\
	x_{1} = x \\
	y_1 = y
\end{array}
\right .
\iff
\frac{\partial y}{\partial x} = \frac{\partial y_{1}}{\partial x} + \frac{\partial y_{1}}{\partial x_{1}}
\]
Why not in $y = x^{x}$ not setting up $\alpha = \{x,x\}$ ? Yes, it's a great idea. Then goes more global form of total derivatives (or partial). 
\\
We can set up this $\alpha_{0} : \alpha_{1}$. It meen, does $\alpha_{0}$ same as $\alpha_{1}$ as object (and of course as number because $x=x$). Then we can rewrite partial derivative of $\{\ y=x^{x} \ ; \ \alpha = \{x,x\}\ \}$
\[
\frac{\partial y}{\partial x} = \sum_{i=0}^{card(\alpha)}
\left \{
\begin{array}{r c l}
	0 \ \ if \ \  \alpha_{i} : x \\
	\frac{\partial y}{\partial \alpha_{i}} \ \ otherwise
\end{array}
\right .
\]

\end{document}

